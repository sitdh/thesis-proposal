\section{ที่มาและความสำคัญ} 

% การทดสอบซอฟต์แวร์เป็นสิ่งที่จำเป็น และขาดไม่ได้
ในกระบวนการพัฒนาซอฟต์แวร์ทรัพยากรมากกว่าครึ่งหนึ่งของการดำเนินโครงการทั้งงบประมาณ เครื่องมือ และบุคลากร นั้นใช้ไปในขั้นตอนการทดสอบซอฟต์แวร์ %\cite 
ทั้งนี้ก็เพื่อตรวจสอบข้อผิดพลาดที่เกิดระหว่างการพัฒนาทั้งในชุดรหัสต้นฉบับ (Source code) ฮาร์ดแวร์ประกอบ รวมไปถึงสภาพแวดล้อมการดำเนินงาน 
จุดประสงค์หลักก็เพื่อรับประกันคุณภาพของซอฟต์แวร์ที่พัฒนาขึ้น และรับรองการทำงานว่าสอดคล้องตามข้อกำหนดที่ระบุไว้

ซึ่งกระบวนการพัฒนาซอฟต์แวร์เองได้มีการประประยุกต์วิธีการให้เข้ากับความต้องการเชิงธุรกิจที่มักเปลี่ยนแปลงอย่างรวดเร็ว ... ดังนั้นการ

% การทำ Regression testing นั้นเป็นสิ่งที่ควรทำ ในการพัฒนา

% การเจอข้อผิดพลาดเร็วยิ่งดี

% การจัดเรียงกรณีทดสอบในช่วงการทำ Regression testing นั้นช่วยให้พบข้อผิดพลาดเร็วยิ่งขึ้น



% การทดสอบซอฟต์แวร์เป็นกระบวนการสำคัญในขั้นตอนการพัฒนาซอฟต์แวร์ เพื่อให้มั่นใจได้ว่าซอฟต์แวร์นั้นสามารถทำงานได้ถูกต้องตามที่กำหนดไว้ \cite{Luo2001} 
% ทั้งยังเป็นขั้นตอนที่ใช้บุคลากร งบประมาณ และระยะเวลาโดยรวมแล้วกว่าครึ่งหนึ่งของทรัพยากรทั้งหมดที่ใช้ในกระบวนการพัฒนาซอฟต์แวร์ \cite{singh2011} 
% และด้วยการพัฒนาโดยใช้ภาษาในยุคที่ 4 (Fourth-generation languages - 4GLs) ที่ต้องการความรวดเร็วในกระบวนการพัฒนาแล้วด้วยนั้น ส่งผลให้ความสำคัญของกระบวนการทดสอบซอฟต์แวร์นั้นเพิ่มมากขึ้นตามไปด้วยเช่นกัน \cite{Luo2001} 

% งานวิจัยนี้มีจุดประสงค์เพื่อนำเสนอวิธีการสร้างกรณีทดสอบตามเส้นทางการเรียกใช้งานระหว่างกันของคลาสที่พัฒนาขึ้นจากภาษาจาวา (Java) 
% โดยอาศัยการวิเคราะห์เส้นทางจาก{\scg} ({\scgEN}) ที่สร้างขึ้นจาก{\sourcecode} ซึ่งรับข้อมูลจาก{\Repository} 
% โดยอาศัยการเก็บข้อมูลในช่วง{\RegressionTesting} ({\RegressionTestingEN}) โดยกรณีทดสอบนั้นครอบคลุมการเรียกใช้ระหว่างกัน
% ตาม{\scg}ที่สร้างขึ้นจาก
งานวิจัยนี้มีจุดประสงค์ที่จะนำเสนอวิธีการสร้างกรณีทดสอบสำหรับ{\sourcecode}ของ{\softwareComponent}ที่นำมาทดสอบเพื่อประกอบเข้าด้วยกัน
ในช่วงการ\FirstTimeDefine{\IntegrationTesting}{\IntegrationTestingEN} ด้วยการสร้างกรณีทดสอบตามเส้นทางการทดสอบ
ที่มีน้ำหนักมากที่สุดจากเส้นทางทั้งหมดของ{\scg} ซึ่งได้จากการวิเคราะห์แบบสถิตของ{\sourcecode}ที่นำมาพิจารณา 
โดยการสร้างกรณีทดสอบนี้จะจัดเตรียมข้อมูลทดสอบจากการพิจารณาข้อมูลค่าคงที่ ที่จัดเก็บจาก{\sourcecode} ได้แก่ ข้อความ ตัวเลข และชุดข้อมูล 
เข้าร่วมกับเงื่อนไขของเส้นทาง (Path conditions) จาก{\cfg} ร่วมกันเพื่อกำหนดขอบเขตการสุ่มค่าข้อมูลนำเข้าที่สอดคล้องกับ{\MethodSignature}ที่ต้องการทดสอบ

