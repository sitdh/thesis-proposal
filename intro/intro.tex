\section{ที่มาและความสำคัญ} 
\label{sec:introduction}

% ในกระบวนการพัฒนาซอฟต์แวร์นั้น ขั้นตอนการทดสอบซอฟต์แวร์ถือเป็นช่วงที่ใช้ทรัพยากรในการดำเนินงานมากที่สุดกว่ากึ่งหนึ่งของทรัพยากรที่ใช้สำหรับพัฒนาซอฟต์แวร์ทั้งหมด 
% ทำให้เห็นได้ว่าการพัฒนาซอฟต์แวร์นั้นเป็นให้ความสำคัญต่อคุณภาพของซอฟต์แวร์เป็นอย่างมาก 
% เพราะการทดสอบซอฟต์แวร์คือขั้นตอนที่ทำขึ้นเพื่อค้นหาข้อผิดพลาดที่อาจยังคงอยู่ในซอฟต์แวร์ ที่อาจจังคงอยู่ภายในซอฟต์แวร์ \cite{Myers:2011:AST:983238} 
% นอกจากนั้นแล้วยังเป็นกระบวนการที่ทำให้สามารถมั่นใจได้ถึงคุณภาพและประสิทธิภาพของซอฟต์แวร์ที่พัฒนาขึ้นด้วย

% การค้นหาข้อผิดพลาดที่ปรากฎอยู่ภายในซอฟต์แวร์นั้น ก็จำเป็นจะต้องสร้างกรณีทดสอบให้ครอบคลุม{\sourcecode}ที่ได้พัฒนาขึ้นให้มากที่สุดเท่าที่จะเป็นไปได้ หลายงานวิจัย [x,x,x,x]

% การทดสอบซอฟต์แวร์เป็นกระบวนการสำคัญในขั้นตอนการพัฒนาซอฟต์แวร์ เพื่อให้มั่นใจได้ว่าซอฟต์แวร์นั้นสามารถทำงานได้ถูกต้องตามที่กำหนดไว้ \cite{Luo2001} 
% ทั้งยังเป็นขั้นตอนที่ใช้บุคลากร งบประมาณ และระยะเวลาโดยรวมแล้วกว่าครึ่งหนึ่งของทรัพยากรทั้งหมดที่ใช้ในกระบวนการพัฒนาซอฟต์แวร์ \cite{singh2011} 
% และด้วยการพัฒนาโดยใช้ภาษาในยุคที่ 4 (Fourth-generation languages - 4GLs) ที่ต้องการความรวดเร็วในกระบวนการพัฒนาแล้วด้วยนั้น ส่งผลให้ความสำคัญของกระบวนการทดสอบซอฟต์แวร์นั้นเพิ่มมากขึ้นตามไปด้วยเช่นกัน \cite{Luo2001} 

% งานวิจัยนี้มีจุดประสงค์เพื่อนำเสนอวิธีการสร้างกรณีทดสอบตามเส้นทางการเรียกใช้งานระหว่างกันของคลาสที่พัฒนาขึ้นจากภาษาจาวา (Java) 
% โดยอาศัยการวิเคราะห์เส้นทางจาก{\scg} ({\scgEN}) ที่สร้างขึ้นจาก{\sourcecode} ซึ่งรับข้อมูลจาก{\Repository} 
% โดยอาศัยการเก็บข้อมูลในช่วง{\RegressionTesting} ({\RegressionTestingEN}) โดยกรณีทดสอบนั้นครอบคลุมการเรียกใช้ระหว่างกัน
% ตาม{\scg}ที่สร้างขึ้นจาก

% การพัฒนาโปรแกรมนั้นจะแยกส่วนประกอบของโปรแกรมออกเป็นส่วนย่อยเพื่อความสะดวกในการพัฒนา ซึ่งในขั้นตอนนี้จะใช้การทดสอบส่วนย่อย (Unit Test) 
% เพื่อหาข้อผิดพลาดที่อยู่โปรแกรม โดยจะจำลองการเชื่อมต่อกับส่วนอื่น ๆ ในรูปแบบของสตับ (Stub) และไดรเวอร์ (Drive) เพื่อจำกัดขอบเขตของการค้นหาข้อผิดพลาด
% หากแต่ในขั้นตอนผนวกรวม{\sourcecode}ของส่วนย่อยเข้าด้วยกันนั้น อาจจะพบข้อผิดพลาดที่เกิดขึ้นบน{\Path}ระหว่าง{\class}ที่อยู่ต่างส่วนย่อยได้

% งานวิจัยนี้มีจุดประสงค์ที่จะนำเสนอวิธีการสร้างกรณีทดสอบสำหรับ{\sourcecode}ของ{\softwareComponent}ที่นำมาทดสอบเพื่อประกอบเข้าด้วยกัน
% ในช่วงการ\FirstTimeDefine{\IntegrationTesting}{\IntegrationTestingEN} ด้วยการสร้างกรณีทดสอบตามเส้นทางการทดสอบ
% ที่มีน้ำหนักมากที่สุดจากเส้นทางทั้งหมดของ{\scg} ซึ่งได้จากการวิเคราะห์แบบสถิตของ{\sourcecode}ที่นำมาพิจารณา 
% โดยการสร้างกรณีทดสอบนี้จะจัดเตรียมข้อมูลทดสอบจากการพิจารณาข้อมูลค่าคงที่ ที่จัดเก็บจาก{\sourcecode} ได้แก่ ข้อความ ตัวเลข และชุดข้อมูล 
% เข้าร่วมกับเงื่อนไขของเส้นทาง (Path conditions) จาก{\cfg} 
% ร่วมกันเพื่อกำหนดขอบเขตการสุ่มค่าข้อมูลนำเข้าที่สอดคล้องกับ\FirstTimeDefine{\MethodSignature}{\MethodSignatureEN}ที่ต้องการทดสอบ

การทดสอบซอฟต์แวร์เป็นขั้นตอนที่ทำขึ้นเพื่อค้นหาข้อผิดพลาดที่ยังคงอยู่ภายในซอฟต์แวร์ \cite{Myers:2011:AST:983238} 
อีกทั้งยังเป็นกระบวนการสร้างความมั่นใจด้านคุณภาพของซอฟต์แวร์ที่ได้พัฒนาขึ้น ดังนั้นกว่ากึ่งหนึ่งของทรัพยากรที่ใช้ในการดำเนินงานโครงการพัฒนาซอฟต์แวร์
จึงถูกจัดสรรค์ให้กับขั้นตอนการทดสอบซอฟต์แวร์นี้ \cite{Jackson2007, Tassey2002} ดังนั้นการทดสอบซอฟต์แวร์จึงถือเป็นขั้นตอนที่สำคัญ
ในกระบวนการพัฒนาซอฟต์แวร์อีกขั้นตอนหนึ่ง

การพัฒนาซอฟต์แวร์นั้นมีแนวทางการพัฒนาหลากหลายแนวทางด้วยกัน ไม่ว่าจะเป็นการกำหนดแนวทางการพัฒนาด้วยการทดสอบ (Test-Driven Development) 
ที่จะสร้างกรณีทดสอบย่อย (Unit Testing) เพื่อกำหนดพฤติกรรมการทำงานของคลาสที่สนใจ หรือ คลาสภายใต้การทดสอบ (Class under test) [x]
และการพัฒนาโดยใช้พฤติกรรมการใช้งาน (Behaviour-Driven Development) ซึ่งจะกำหนดลักษณะการใช้งานซอฟต์แวร์แล้ว
จึงพัฒนาส่วนประกอบของซอฟต์แวร์ที่สอดคล้องกับการใช้งานนั้น [x] 
ซึ่งทั้ง 2 แนวทางนั้นจะสร้างกรณีทดสอบ เพื่อใช้ในการทดสอบส่วนย่อย (Unit Testing) โดยจะจำลองพฤติกรรมการทำงานของคลาสหรือส่วนประกอบอื่น ๆ
(Mocking) เช่น ฐานข้อมูล เว็บเซอร์วิส หรืออุปกรณ์เชื่อมต่อที่ต้องการ ที่มีปฏิสัมพันธ์กับคลาสภายใต้การทดสอบ ในลักษณะของ (Stub) และไดรเวอร์ (Driver)
เพื่อจำกัดขอบเขตของข้อผิดพลาดให้อยู่เฉพาะแต่เพียงคลาสภายใต้การทดสอบเท่านั้น แต่เพื่อให้มั่นใจได้ว่าซอฟต์แวร์สามารถทำงานร่วมกันได้ 
ตามพฤติกรรมที่ได้จำลองไว้สตับและไดรเวอร์ในขั้นตอนการการทดสอบย่อยนั้น 
จึงจำเป็นจะต้องทดสอบการทำงานร่วมกันด้วย\FirstTimeDefine{\IntegrationTesting}{\IntegrationTestingEN}
โดยในขั้นตอนนี้ จะใช้การเชื่อมต่อไปยังส่วนประกอบที่มีปฏิสัมพันธ์กับกระบวนการนั้นโดยตรง เพื่อให้มั่นใจได้ว่าซอฟต์แวร์ภายใต้การทดสอบ (Software under test) 
สามารถดำเนินงานตามกระบวนการที่สนใจได้โดยไม่เกิดข้อขัดข้อง

ทั้งนี้ การสร้างกรณีทดสอบในขั้นตอนก{\IntegrationTesting}นั้นจะสร้างกรณีทดสอบสอดคล้องตามกระบวนการทำงานของ{\SUT}ที่สนใจ 
โดยผลลัพธ์ที่เป็นไปได้ย่อมเกิดระหว่างการเรียกใช้งานระหว่างส่วนประกอบภายในซอฟต์แวร์{\SUT}ด้วยกันเอง ตลอดจนส่วนประกอบภายนอก 
ซึ่งล้วนแล้วแต่มีหลายปัจจัยที่ต้องนำมาพิจารณาร่วมกัน การสร้างกรณีทดสอบอัตโนมัติจึงเข้ามามีบทบาทสำคัญ ช่วยให้การสร้างกรณีทดสอบนั้นขั้นตอนนี้ทำได้รวดเร็วมากยิ่งขึ้น 
หากแต่การสร้างกรณีทดสอบในขั้นตอน{\IntegrationTesting} ยังต้องใช้\FirstTimeDefine{\DynamicInformation}{\DynamicInformationEN}
ซึ่งได้จากการสั่งกระทำการ (Execute) ระหว่างชุดกรณีทดสอบกับ{\sourcecode} ซึ่งข้อมูลที่ได้จากวิธีการนี้สามารถอธิบายพฤติกรรมของ{\SUT}ของ{\sourcecode}
ได้เป็นอย่างดี แต่หากเกิดการเปลี่ยนแปลงของ{\sourcecode}ภายหลังจากที่ได้สั่งกระทำการชุดทดสอบไปแล้ว 
จะทำให้ชุดทดสอบนั้นไม่สามารถอธิบายคุณลักษณะของ{\sourcecode}ที่เกิดการเปลี่ยนแปลงได้ ดังนั้นหากสามารถนำข้อมูลเชิงสถิตที่ได้จาก{\sourcecode}
มาสร้างชุดทดสอบได้ ก็จะทำให้ชุดทดสอบที่ได้สามารถอธิบายคุณลักษณะตลอดจนการดำเนินงานของ{\SUT}ที่เป็นปัจจุบันได้

งานวิจัยนี้มีจุดประสงค์ที่จะนำเสนอวิธีการสร้างกรณีทดสอบสำหรับ{\sourcecode}ของ{\softwareComponent}ที่นำมาทดสอบเพื่อประกอบเข้าด้วยกัน
ในช่วงการ\FirstTimeDefine{\IntegrationTesting}{\IntegrationTestingEN} ด้วยการสร้างกรณีทดสอบตามเส้นทางการทดสอบ
ที่มีน้ำหนักมากที่สุดจากเส้นทางทั้งหมดของ{\scg} ซึ่งได้จากการวิเคราะห์แบบสถิตของ{\sourcecode}ที่นำมาพิจารณา 
โดยการสร้างกรณีทดสอบนี้จะจัดเตรียมข้อมูลทดสอบจากการพิจารณาข้อมูลค่าคงที่ ที่จัดเก็บจาก{\sourcecode} ได้แก่ ข้อความ ตัวเลข และชุดข้อมูล 
เข้าร่วมกับเงื่อนไขของเส้นทาง (Path conditions) จาก{\cfg} 
ร่วมกันเพื่อกำหนดขอบเขตการสุ่มค่าข้อมูลนำเข้าที่สอดคล้องกับ\FirstTimeDefine{\MethodSignature}{\MethodSignatureEN} ที่ต้องการทดสอบ
