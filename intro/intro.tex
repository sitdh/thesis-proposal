\section{ที่มาและความสำคัญ} 
\label{sec:introduction}

การทดสอบซอฟต์แวร์เป็นขั้นตอนที่ทำขึ้นเพื่อค้นหาข้อผิดพลาดที่ยังคงอยู่ภายในซอฟต์แวร์ \cite{Myers:2011:AST:983238}
อีกทั้งยังเป็นกระบวนการสร้างความมั่นใจด้านคุณภาพของซอฟต์แวร์ที่ได้พัฒนาขึ้น ดังนั้นกว่ากึ่งหนึ่งของทรัพยากรที่ใช้ในการดำเนินงานโครงการพัฒนาซอฟต์แวร์
จึงถูกจัดสรรค์ให้กับขั้นตอนการทดสอบซอฟต์แวร์นี้ \cite{Jackson2007, Tassey2002} ดังนั้นการทดสอบซอฟต์แวร์จึงถือเป็นขั้นตอนที่สำคัญ
ในกระบวนการพัฒนาซอฟต์แวร์อีกขั้นตอนหนึ่ง

การพัฒนาซอฟต์แวร์นั้นมีแนวทางการพัฒนาหลากหลายแนวทางด้วยกัน ไม่ว่าจะเป็นการกำหนดแนวทางการพัฒนาด้วยการทดสอบ (Test-Driven Development) 
ที่จะสร้างกรณีทดสอบย่อย (Unit Testing) เพื่อกำหนดพฤติกรรมการทำงานของคลาสที่สนใจหรือคลาสภายใต้การทดสอบ (Class under test) 
ก่อนที่จะเริ่มพัฒนาซอฟต์แวร์จริง \cite{Markovi2012} และการพัฒนาโดยใช้พฤติกรรมการใช้งาน (Behaviour-Driven Development) 
ซึ่งกำหนดลักษณะการใช้งานซอฟต์แวร์แล้ว จึงพัฒนาส่วนประกอบของซอฟต์แวร์ที่สอดคล้องกับการใช้งานนั้น \cite{Lazar2010}
ซึ่งทั้ง 2 แนวทางนั้นจะสร้างกรณีทดสอบ เพื่อใช้ในการทดสอบส่วนย่อย (Unit Testing) โดยจะจำลองพฤติกรรมการทำงานของคลาสหรือส่วนประกอบอื่น ๆ
(Mocking) เช่น ฐานข้อมูล เว็บเซอร์วิส หรืออุปกรณ์เชื่อมต่อที่ต้องการ ที่มีปฏิสัมพันธ์กับคลาสภายใต้การทดสอบ ในลักษณะของ (Stub) และไดรเวอร์ (Driver)
เพื่อจำกัดขอบเขตของข้อผิดพลาดให้อยู่เฉพาะแต่เพียงคลาสภายใต้การทดสอบเท่านั้น แต่เพื่อให้มั่นใจได้ว่าซอฟต์แวร์สามารถทำงานร่วมกันได้ 
ตามพฤติกรรมที่ได้จำลองไว้สตับและไดรเวอร์ในขั้นตอนการการทดสอบย่อยนั้น 
จึงจำเป็นจะต้องทดสอบการทำงานร่วมกันด้วย\FirstTimeDefine{\IntegrationTesting}{\IntegrationTestingEN}
โดยในขั้นตอนนี้ จะใช้การเชื่อมต่อไปยังส่วนประกอบที่มีปฏิสัมพันธ์กับกระบวนการนั้นโดยตรง เพื่อให้มั่นใจได้ว่าซอฟต์แวร์ภายใต้การทดสอบ (Software under test) 
สามารถดำเนินงานตามกระบวนการที่สนใจได้โดยไม่เกิดข้อขัดข้อง

ทั้งนี้ การสร้างกรณีทดสอบในขั้นตอนก{\IntegrationTesting}นั้นจะสร้างกรณีทดสอบสอดคล้องตามกระบวนการทำงานของ\FirstTimeDefine{\SUT}{\SUTEN} 
โดยผลลัพธ์ที่เป็นไปได้ย่อมเกิดระหว่างการเรียกใช้งานระหว่าง{\softwareComponent}ภายใน{\SUT}ด้วยกันเอง ตลอดจนส่วนประกอบภายนอก 
ซึ่งล้วนแล้วแต่มีหลายปัจจัยที่ต้องนำมาพิจารณาร่วมกัน การสร้างกรณีทดสอบอัตโนมัติจึงมีบทบาทสำคัญ สำหรับช่วยให้การสร้างกรณีทดสอบทำได้รวดเร็วมากยิ่งขึ้น 
หากแต่การสร้างกรณีทดสอบในขั้นตอน{\IntegrationTesting} ยังต้องใช้\FirstTimeDefine{\DynamicInformation}{\DynamicInformationEN}
ได้จากการสั่งกระทำการ (Execute) ระหว่างชุดกรณีทดสอบกับ{\sourcecode} โดยข้อมูลที่ได้จากวิธีการนี้สามารถอธิบายพฤติกรรมของ{\SUT}ของ{\sourcecode}
ได้เป็นอย่างดี หากแต่การเปลี่ยนแปลงของ{\sourcecode}นั้นอากเกิดขึ้นภายหลังจากที่ได้สั่งกระทำการชุดทดสอบไปแล้ว 
ส่งผลให้ชุดทดสอบนั้นไม่สามารถอธิบายคุณลักษณะของ{\sourcecode}ที่เป็นปัจจุบันได้ ดังนั้นหากสามารถนำ\FirstTimeDefine{\StaticInformation}{\StaticInformationEN}
ที่ได้จาก{\sourcecode}ที่เป็นปัจจุบัน มาใช้เป็นข้อมูลในขั้นตอนการสร้างชุดทดสอบให้ครอบคลุมตามความสัมพันธ์ระหว่าง{\softwareComponent}ซึ่งปรากฎใน{\SUT} 
จะทำให้ชุดทดสอบนั้น สามารถอธิบายคุณลักษณะตลอดจนการดำเนินงานของ{\SUT}ที่เป็นปัจจุบันได้

ดังนั้น การวิจัยนี้จึงต้องการที่จะนำเสนอแนวทางการสร้างกรณีทดสอบ บน{\TestPath}ซึ่งพิจารณาจากความสัมพันธ์ระหว่าง{\softwareComponent} 
ที่ด้วยรวบรวมได้จากข้อมูล{\StaticInformation}ของ{\sourcode}ภายใน{\Package}ภายใน{\SUT} พร้อมด้วยการสร้างข้อมูลทดสอบ (Test data) 
ที่สอดคล้องตามเงื่อนไขที่ปรากฏอยู่บน{\TestPath} ที่เลือก ทั้งนี้เพื่อให้ชุดทดสอบที่สร้างขึ้นสามารถทดสอบแต่ละ{\Path}ระหว่าง{\softwareComponent}อย่างน้อย 1 ครั้ง
โดยใช้ผลลัพธ์ที่เกิดจากการสั่งกระทำการชุดกรณีทดสอบที่สร้างขึ้นกับ{\sourcecode}ซึ่งผ่านการแทรกชุดคำสั่งสำหรับติดตามการทำงาน 
เพื่อตรวจสอบว่า กรณีทดสอบนั้นสามารถทดสอบทุกความสัมพันธ์ระหว่าง{\softwareComponent}ได้ตามที่ต้องการ
