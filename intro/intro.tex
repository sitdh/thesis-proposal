\section{ที่มาและความสำคัญ} 
\label{sec:introduction}

การทดสอบซอฟต์แวร์เป็นขั้นตอนที่ทำขึ้นเพื่อค้นหาข้อผิดพลาดที่ยังคงอยู่ภายในซอฟต์แวร์ \cite{Myers:2011:AST:983238} 
อีกทั้งยังเป็นกระบวนการสร้างความเชื่อมั่นด้านคุณภาพของซอฟต์แวร์ที่ได้พัฒนาขึ้น ดังนั้นกว่ากึ่งหนึ่งของทรัพยากรที่ใช้ในการดำเนินงานโครงการพัฒนาซอฟต์แวร์
จึงถูกจัดสรรค์ให้กับขั้นตอนการทดสอบซอฟต์แวร์นี้ \cite{Jackson2007, Tassey2002} ดังนั้นการทดสอบซอฟต์แวร์จึงถือเป็นขั้นตอนที่สำคัญ
ในกระบวนการพัฒนาซอฟต์แวร์อีกขั้นตอนหนึ่ง

การสร้าวกรณีทดสอบสำหรับ\FirstTimeDefine{\SUT}{\SUTEN} นั้นมีหลากหลายวิธีการขึ้นอยู่กับเป้าหมายของการดำเนินงานนั้น ๆ 
ทั้งการสร้างกรณีทดสอบเพื่อตรวจสอบว่า{\method}ที่อยู่ภายในคลาสนั้นถูกเรียกใช้งานครบถ้วนแล้วหรือไม่ \cite{6847360}
หรือสร้างกรณีทดสอบเพื่อให้ครบคลุมเงื่อนไขการตัดสินใจที่ปรากฏภายใน{\class} \cite{Luanghirun2016} ซึ่งเป็นการพิจารณาเพียงเฉพาะในคลาสเท่านั้น
หากแต่เมื่อต้องการทดสอบการทำงานร่วมกันของ\FirstTimeDefine{\softwareComponent}{\softwareComponentEN} 
ภายใน\FirstTimeDefine{\SUT}{\SUTEN} กรณีทดสอบที่สร้างขึ้นนั้นก็จำเป็นจะต้องครอบคลุมการเรียกใช้งานระหว่างกันของ{\softwareComponent}ที่พัฒนาขึ้น
ซึ่งมีแนวทางการสร้างกรณีทดสอบในลักษณะนี้หลายแนวทาง อาทิ 
การสร้างกรณีทดสอบโดยอาศัยการพิจารณาแผนภาพอธิบายโครงสร้างหรือพฤติกรรมของ{\softwareComponent} \cite{Panthi2012, Shirole2013}
โดยผลลัพธ์ที่เป็นไปได้ย่อมเกิดระหว่างการเรียกใช้งานระหว่างส่วนประกอบภายในซอฟต์แวร์{\SUT} ด้วยกันเอง 
ตลอดจนส่วนประกอบภายนอก ซึ่งล้วนแล้วแต่มีหลายปัจจัยที่ต้องนำมาพิจารณาร่วมกัน เพื่อให้มั่นใจได้ว่าซอฟต์แวร์ภายใต้การทดสอบ (Software under test) 
สามารถดำเนินงานตามกระบวนการที่สนใจได้โดยไม่เกิดข้อขัดข้อง

การสร้างกรณีทดสอบอัตโนมัติจึงเข้ามามีบทบาทสำคัญ ช่วยให้การสร้างกรณีทดสอบนั้นขั้นตอนนี้ทำได้รวดเร็วมากยิ่งขึ้น 
หากแต่การสร้างกรณีทดสอบนั้นยังต้องใช้\FirstTimeDefine{\DynamicInformation}{\DynamicInformationEN}
ซึ่งได้จากการสั่งกระทำการ (Execute) ระหว่างชุดกรณีทดสอบกับ{\sourcecode} ซึ่งข้อมูลที่ได้จากวิธีการนี้สามารถอธิบายพฤติกรรมของ{\SUT}ของ{\sourcecode}
ได้เป็นอย่างดี แต่หากเกิดการเปลี่ยนแปลงของ{\sourcecode}ภายหลังจากที่ได้สั่งกระทำการชุดทดสอบไปแล้ว 
จะทำให้ชุดทดสอบนั้นไม่สามารถอธิบายคุณลักษณะของ{\sourcecode}ที่เกิดการเปลี่ยนแปลงได้ ดังนั้นหากสามารถนำข้อมูลเชิงสถิตที่ได้จาก{\sourcecode}
มาสร้างชุดทดสอบได้ ก็จะทำให้ชุดทดสอบที่ได้สามารถอธิบายคุณลักษณะตลอดจนการดำเนินงานของ{\SUT}ที่เป็นปัจจุบันได้

งานวิจัยนี้มีจุดประสงค์ที่จะนำเสนอวิธีการสร้างกรณีทดสอบสำหรับ{\sourcecode}ของ{\softwareComponent}ที่นำมาทดสอบเพื่อประกอบเข้าด้วยกัน
ในช่วงการ\FirstTimeDefine{\IntegrationTesting}{\IntegrationTestingEN} ด้วยการสร้างกรณีทดสอบตามเส้นทางการทดสอบครอบคลุม
{\scg} ซึ่งได้จากการวิเคราะห์\FirstTimeDefine{\StaticInformation}{\StaticInformationEN} ของ{\sourcecode}ที่นำมาพิจารณา 
โดยการสร้างกรณีทดสอบนี้จะจัดเตรียมข้อมูลทดสอบจากการพิจารณาข้อมูลค่าคงที่ ที่จัดเก็บจาก{\sourcecode} ได้แก่ ข้อความ ตัวเลข และชุดข้อมูล 
เข้าร่วมกับเงื่อนไขของเส้นทาง (Path conditions) จาก{\cfg} 
ร่วมกันเพื่อกำหนดขอบเขตการสุ่มค่าข้อมูลนำเข้าที่สอดคล้องกับ\FirstTimeDefine{\MethodSignature}{\MethodSignatureEN} ที่ต้องการทดสอบ
