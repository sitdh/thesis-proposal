\section{ขอบเขตงานวิจัย}
\label{sec:limitation}

\begin{enumerate}
    \label{enu:limitation}
    \item สร้างกรณีทดสอบที่ครอบคลุม{\TestPath}ระหว่าง{\CUT} \label{enu:lim:tc}
    \item รองรับการเรียกใช้งานกันของ{\CUT}บน{\TestPath}สูงสุด 3 {\class} \label{enu:lim:3linkingclass}
    \item รองรับจำนวน{\method}ภายใน{\CUT}สูงสุด{\class}ละไม่เกิน 5 {\method} \label{enu:lim:5methods}
    % \item รองรับ{\class}ภาษาจาวาขนาดความยาวไม่เกิน 200 บรรทัดต่อ{\class} \label{enu:lim:200loc} 
        % http://softwareengineering.stackexchange.com/questions/66523/how-many-lines-per-class-is-too-many-in-java
    % \item รองรับ{\TestPath}ที่ปรากฎ{\PredicateNode}ตลอดทั้ง{\Path}ไม่เกิน 20 {\Node}  \label{enu:lim:20predicate}
    \item ข้อมูล{\scg}ที่สร้างขึ้นนั้นจะสนใจเฉพาะความสัมพันธ์ระหว่างคลาสที่อยู่ภายใน{\Package}จาวา (Java package) ตามที่ผู้ใช้ระบุ 
        โดยละเว้นชุดพัฒนามาตรฐานของภาษาจาวาและชุดคำสั่งภายนอก \label{enu:lim:scg}
    \item รองรับข้อมูลนำเข้าประเภทตัวอักษร (มีชนิดข้อมูลเป็น \code{String} ในภาษาจาวา) ตัวเลข 
        (มีชนิดข้อมูลเป็น \code{byte, short, int, long, float} และ \code{double} ในภาษาจาวา) และ{\enum} 
        (มีชนิดข้อมูลเป็น \code{enum} ในภาษาจาวา) ยังไม่รองรับประเภทข้อมูลเฉพาะที่สร้างขึ้นเอง \label{enu:lim:datatype}
    \item รองรับโปรแกรมประยุกต์ที่พัฒนาขึ้นเพื่อใช้งานบนเครื่องคอมพิวเตอร์ส่วนบุคคล และ{\sourcecode}พัฒนาขึ้นด้วยภาษาจาวา  \label{enu:lim:datatype}
        % โดยไม่รวมถึงโปรแกรมประยุกต์ประเภทเว็บแอปพลิเคชัน
    \item ครอบคลุมโปรแกรมที่ทำงานเป็นลำดับ (Sequential) ไม่ครอบคลุมการทำงานของโปรแกรมที่ทำงานเป็นภาวะพร้อมกัน (Parallel) \label{enu:lim:seq}
    \item รองรับเฉพาะโปรแกรมที่มีจำนวนของไซโคลเมทิกภายในคลาสรวมกันแล้วน้อยกว่าหรือเท่ากับ 15
    \item รองรับการสร้างกรณีทดสอบที่สอดคล้องตามรูปแบบของชุดพัฒนา JUnit
    \item รองรับการดำเนินงานแบบวงวน (Loop) มากที่สุด 1 รอบ \label{enu:lim:loop}
    \item ไม่รองรับการเรียกใช้ฟังก์ชันเวียนบังเกิด (Recursion)
\end{enumerate}
