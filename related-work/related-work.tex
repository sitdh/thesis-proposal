\section{งานวิจัยที่เกี่ยวข้อง} 

ในการดำเนินการวิจัยครั้งนี้ ได้ศึกษางานวิจัยอื่น ๆ ที่คาดว่าจะสามารถนำมาประยุกต์ใช้เพื่อแก้ไขปัญหาข้างต้นได้ โดยมีงานวิจัยที่เลือกมาดังนี้

% - - - - - - - - - - - - - - - - - - - -
\subsection{งานวิจัย "A Static Approach to Prioritizing JUnit Test Cases" \cite{6363461}}

งานวิจัยชิ้นนี้ได้นำเสนอวิธีการจัดเรียงกรณีทดสอบด้วยแนวทาง \emph{"JUnit test case Prioritization Techniques operating in the Absence of coverage information (JUPTA)"} 
โดยอาศัยการวิเคราะห์ข้อมูล{\scg}ที่สร้างขึ้นจากกรณีทดสอบ และโปรแกรมภายใต้การทดสอบเพื่อประมาณค่าความสามารถที่กรณีทดสอบนั้นสามารถครอบคลุม{\sourcecode}ได้
ทั้งนี้เพื่อเป็นการพิสูจน์แนวคิดที่ผู้วิจัยพบว่าวิธีการจัดเรียงกรณีทดสอบโดยส่วนใหญ่ที่ทราบจากการทบทวนวรรณกรรมนั้น จะใช้ข้อมูลเชิงพลวัต (Dynamic information)
ที่ได้จากการทดสอบ{\sourcecode}ในแต่ละครั้ง เข้ามาช่วยวิเคราะห์ลำดับของกรณีทดสอบในครั้งถัดไป พบว่าข้อมูลที่ได้นั้นอาจจะเป็นข้อมูลที่ล้าหลังไปจาก{\sourcecode} 
หรือชุดกรณีทดสอบที่สร้างขึ้นมาใหม่เพื่อทดสอบในรอบปัจจุบัน 

ผลการทดสอบทางสถิติของการจัดเรียงกรณีทดสอบด้วยแนวคิด JUPTA ด้วยการจัดเรียงกรณีทดสอบของโปรแกรมภาษาจาวาทั้ง 4 โปรแกรม ที่มีขนาดใกล้เคียงกัน
รวมกว่า 19 รุ่นนั้นพบว่า กรณีทดสอบที่จัดเรียงด้วยแนวคิด JUPTA นั้นมีประสิทธิภาพในการค้นหาข้อผิดพลาดมากกว่าการจัดเรียงกรณีทดสอบแบบสุ่มหรือไม่มีการจัดเรียงใด ๆ 
และมีประสิทธิภาพในการค้นหาข้อผิดพลาดใกล้เคียงกันกับชุดกรณีทดสอบ จัดเรียงโดยอาศัยข้อมูลเชิงพลวัตประกอบการพิจารณา

% - - - - - - - - - - - - - - - - - - - -
\subsection{งานวิจัย "GRT: Program-analysis-guided random testing" \cite{Ma2016}}

งานวิจัยชิ้นนี้ได้นำเสนอกระบวนการปรับปรุงการสุ่มค่าที่จากเดิมนั้นจะสุ่มค่าแบบไร้ทิศทาง หรือกำหนดกรอบการสุ่มข้อมูลนำเข้าไว้อย่างกว้าง ๆ 
ทำให้ค่าข้อมูลนำเข้าที่สุ่มได้นั้นมีโอกาสที่จะค้นพบข้อผิดพลาดภายในโปรแกรมได้น้อย งานวิจัยชิ้นนี้จึงได้เสนอแนวทางการนำข้อมูลที่จาก{\sourcecode} ของ
ซอฟต์แวร์ภายใต้การทดสอบเข้ามาร่วมพิจารณา โดยจัดเก็บค่าคงที่ ที่ใช้งานโดยทั่วไป (Global) ภายใน{\sourcecode} เข้าร่วมพิจารณา 

ซึ่งกระบวนการนั้นจะประกอบไปด้วยการจัดเก็บข้อมูลจาก{\sourcecode} และข้อมูลที่ได้จากการสั่งดำเนินการร่วมกัน ซึ่งให้ผลว่าสามารถสร้างกรอบการสุ่มค่าเพื่อช่วยให้
ได้ข้อมูลนำเข้าที่สามารถค้นพบข้อผิดพลาดภายในชุดข้อมูลตัวอย่างได้มากขึ้น เมื่อเทียบกับวิธีการสุ่มข้อมูลแบบเดิม

% - - - - - - - - - - - - - - - - - - - -
\subsection{งานวิจัย "Eclat: Automatic Generation and Classification of Test Inputs" \cite{Heaton2000}}

เนื่องจากการข้อมูลที่จะนำมาทดสอบนั้นมีขนาดใหญ่ ดังนั้นงานวิจัยนี้จึงเสนอแนวทางในการวิเคราะห์เพื่อหาซับเซตของข้อมูลนำเข้าทั้งหมดที่สามารถค้นพบข้อผิดพลาด
ของซอฟต์แวร์ภายใต้การทดสอบได้ ซึ่งงานวิจัยได้นำเสนอวิธีการเลือกกลุ่มข้อมูลนำเข้าโดยพิจารณาจากส่วนย่อยของซอฟต์แวร์ 
ประกอบกับชุดข้อมูลที่ทำให้โปรแกรมทำงานได้ถูกต้อง ซึ่งผลลัพธ์ที่ทดลองกับกลุ่มตัวอย่างได้พบว่าวิธีการนี้สามารถลดชุดข้อมูลนำเข้าที่ใช้ทดสอบได้
โดยยังมีประสิทธิภาพในการค้นหาข้อผิดพลาดไม่ต่างไปจากเดิม

% % - - - - - - - - - - - - - - - - - - - -
% \subsection{งานวิจัย "Automatic generation and classification of test inputs" \cite{Pacheco2005}}
% การสร้างกลุ่มของกรณีทดสอบอย่างอัตโนมัติ ช่วยในการสร้างกรณีทดสอบ
