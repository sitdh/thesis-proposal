\section{งานวิจัยที่เกี่ยวข้อง} 

ในการดำเนินการวิจัยครั้งนี้ ได้ศึกษางานวิจัยอื่น ๆ ที่คาดว่าจะสามารถนำมาประยุกต์ใช้เพื่อแก้ไขปัญหาข้างต้นได้ โดยมีงานวิจัยที่เลือกมาดังนี้

% - - - - - - - - - - - - - - - - - - - -
\subsection{งานวิจัย {\it "A Static Approach to Prioritizing JUnit Test Cases"} \cite{6363461}}

งานวิจัยชิ้นนี้ได้นำเสนอวิธีการจัดเรียงกรณีทดสอบด้วยแนวทาง \emph{"JUnit test case Prioritization Techniques operating in the Absence of coverage information (JUPTA)"} 
โดยอาศัยการวิเคราะห์ข้อมูล{\scg}ที่สร้างขึ้นจากกรณีทดสอบ และโปรแกรมภายใต้การทดสอบเพื่อประมาณค่าความสามารถที่กรณีทดสอบนั้นสามารถครอบคลุม{\sourcecode}ได้
ทั้งนี้เพื่อเป็นการพิสูจน์แนวคิดที่ผู้วิจัยพบว่าวิธีการจัดเรียงกรณีทดสอบโดยส่วนใหญ่ที่ทราบจากการทบทวนวรรณกรรมนั้น จะใช้ข้อมูลเชิงพลวัต (Dynamic information)
ที่ได้จากการทดสอบ{\sourcecode}ในแต่ละครั้ง เข้ามาช่วยวิเคราะห์ลำดับของกรณีทดสอบในครั้งถัดไป พบว่าข้อมูลที่ได้นั้นอาจจะเป็นข้อมูลที่ล้าหลังไปจาก{\sourcecode} 
หรือชุดกรณีทดสอบที่สร้างขึ้นมาใหม่เพื่อทดสอบในรอบปัจจุบัน 

ผลการทดสอบทางสถิติของการจัดเรียงกรณีทดสอบด้วยแนวคิด JUPTA ด้วยการจัดเรียงกรณีทดสอบของโปรแกรมภาษาจาวาทั้ง 4 โปรแกรม ที่มีขนาดใกล้เคียงกัน
รวมกว่า 19 รุ่นนั้นพบว่า กรณีทดสอบที่จัดเรียงด้วยแนวคิด JUPTA นั้นมีประสิทธิภาพในการค้นหาข้อผิดพลาดมากกว่าการจัดเรียงกรณีทดสอบแบบสุ่มหรือไม่มีการจัดเรียงใด ๆ 
และมีประสิทธิภาพในการค้นหาข้อผิดพลาดใกล้เคียงกันกับชุดกรณีทดสอบ จัดเรียงโดยอาศัยข้อมูลเชิงพลวัตประกอบการพิจารณา

% - - - - - - - - - - - - - - - - - - - -
\subsection{งานวิจัย {\it "Intelligent test case generation based on branch and bound"} \cite{XING201491}}
\label{sec:sub:bandb}

การสร้างกรณีทดสอบโดยพิจารณาจาก{\Path}ของโครงสร้างซอฟต์แวร์นั้น เป็นการแก้ไข\FirstTimeDefine{\csp}{\cspEN} มักแก้ไขปัญหาด้วยแนวทางการค้นหา
ด้วยขั้นตอนวิธีติดตามย้อนกลับ (Backtracking algorithm) ซึ่งในงานวิจัยชิ้นนี้ได้นำขั้นตอนวิธีแตกกิ่งและกำหนดขอบเขต (Branch and bound) 
ร่วมกับขั้นตอนวิธีการค้นหาแบบดีที่สุดก่อน (Best-first search) ช่วยในการสร้างกรณีทดสอบแบบอัตโนมัติ จาก{\StaticInformation}ของ{\sourcecode} 
ประกอบไปด้วย 2 แนวทางด้วยกันนั่นคือ 
1) การปรับเปลี่ยนค่าตัวของตัวแปรด้วยกฎการแก้ไขปัญหาใช้ลดขนาดของปริภูมิคำตอบ ตลอดจนกำจัดตัวแปรที่ไม่เกี่ยวข้องกับการสร้างชุดคำตอบ 
เพื่อนำมาใช้ในขั้นตอนการแตกกิ่ง (Branching operation) และ 2) การคำนวณช่วงค่าที่เป็นไปของตัวแปรเพื่อใช้ในขั้นตอนการกำหนดขอบเขต
ซึ่งงานวิจัยนี้ผู้วิจัยได้ปรับปรุงประสิทธิภาพการทำงานจาก \code{O(n^2)} เป็น \code{O(n)} และเมื่อนำไปทดสอบกับชุดข้อมูลทดสอบภาษา C ทั้งหมด 4 โปรแกรม
ผลปรากฎว่า วิธีการนี้สามารถสร้างกรณีทดสอบได้ครอบคลุมเงื่อนไขการดำเนินงานทั้งหมด เมื่อเทียบกับการสร้างกรณีทดสอบด้วย ขั้นตอนวิธีเชิงพันธุกรรม (Genetic algorithm)
หรือ การจำลองการอบเหนียว (simulated annealing) % Ref: https://www.cp.eng.chula.ac.th/~somchai/CD/2110427/2546/demo/HW2-TSP/saTSP.htm

จะเห็นได้จากการทดลองของผู้วิจัยที่สามารถสร้างกรณีทดสอบด้วย{\Algorithm}ที่นำเสนอ สามารถสร้างกรณีทดสอบได้ครอบคลุมมากกว่า{\Algorithm}ที่นำมาเปรียบเทียบ 
โดยใช้เพียง{\StaticInformation}เท่านั้น ซึ่งไม่จำเป็นต้องส่งกระทำการทดสอบ{\sourcecode}แต่อย่างใด หากแต่ข้อมูลทดสอบที่ใช้ในการทดลองนี้มีขนาดเล็ก 
(จำนวนแถวน้อยที่สุดคือ 21 บรรทัด และมากที่สุดคือ 59 บรรทัด) และเป็น{\sourcecode}ที่ไม่ได้อยู่ในลักษณะของการพัฒนาโปรแกรมเชิงวัตถุ 
ดังนั้นหากสามารถนำขั้นตอนวิธีดังที่งานวิจัยนำเสนอมาใช้ในโปรแกรมที่มีขนาดใหญ่ขั้นและพัฒนาด้วยแนวคิดเชิงวัตถุ 
จะทำให้สามารถหากรณีทดสอบที่ครอบคลุมความสัมพันธ์ระหว่าง{\softwareComponent}ได้ดียิ่งขึ้น

% - - - - - - - - - - - - - - - - - - - -
\subsection{งานวิจัย {\it "GRT: Program-analysis-guided random testing"} \cite{Ma2016}}
\label{sec:sub:grt}

งานวิจัยชิ้นนี้ได้นำเสนอกระบวนการปรับปรุงการสุ่มค่าที่จากเดิมนั้นจะสุ่มค่าแบบไร้ทิศทาง หรือกำหนดกรอบการสุ่มข้อมูลนำเข้าไว้อย่างกว้าง ๆ 
ทำให้ค่าข้อมูลนำเข้าที่สุ่มได้นั้นมีโอกาสที่จะค้นพบข้อผิดพลาดภายในโปรแกรมได้น้อย งานวิจัยชิ้นนี้จึงได้เสนอแนวทางการนำข้อมูลที่จาก{\sourcecode} ของ
ซอฟต์แวร์ภายใต้การทดสอบเข้ามาร่วมพิจารณา โดยจัดเก็บค่าคงที่ ที่ใช้งานโดยทั่วไป (Global) ภายใน{\sourcecode} เข้าร่วมพิจารณา 

ซึ่งกระบวนการนั้นจะประกอบไปด้วยการจัดเก็บข้อมูลจาก{\sourcecode} และข้อมูลที่ได้จากการสั่งดำเนินการร่วมกัน ซึ่งให้ผลว่าสามารถสร้างกรอบการสุ่มค่าเพื่อช่วยให้
ได้ข้อมูลนำเข้าที่สามารถค้นพบข้อผิดพลาดภายในชุดข้อมูลตัวอย่างได้มากขึ้น เมื่อเทียบกับวิธีการสุ่มข้อมูลแบบเดิม

% - - - - - - - - - - - - - - - - - - - -
\subsection{งานวิจัย "Eclat: Automatic Generation and Classification of Test Inputs" \cite{Heaton2000}}
\label{sec:sub:eclat}

เนื่องจากการข้อมูลที่จะนำมาทดสอบนั้นมีขนาดใหญ่ ดังนั้นงานวิจัยนี้จึงเสนอแนวทางในการวิเคราะห์เพื่อหาซับเซตของข้อมูลนำเข้าทั้งหมดที่สามารถค้นพบข้อผิดพลาด
ของซอฟต์แวร์ภายใต้การทดสอบได้ ซึ่งงานวิจัยได้นำเสนอวิธีการเลือกกลุ่มข้อมูลนำเข้าโดยพิจารณาจากส่วนย่อยของซอฟต์แวร์ 
ประกอบกับชุดข้อมูลที่ทำให้โปรแกรมทำงานได้ถูกต้อง ซึ่งผลลัพธ์ที่ทดลองกับกลุ่มตัวอย่างได้พบว่าวิธีการนี้สามารถลดชุดข้อมูลนำเข้าที่ใช้ทดสอบได้
โดยยังมีประสิทธิภาพในการค้นหาข้อผิดพลาดไม่ต่างไปจากเดิม

% % - - - - - - - - - - - - - - - - - - - -
% \subsection{งานวิจัย "Automatic generation and classification of test inputs" \cite{Pacheco2005}}
% การสร้างกลุ่มของกรณีทดสอบอย่างอัตโนมัติ ช่วยในการสร้างกรณีทดสอบ
