\pagenumbering{thaialph}
\begin{table}[ht!]
    \centering
    \begin{tabularx}{\textwidth}{|l|l|l|X|}
        \hline
        \rowcolor{LightGray} 
        {\bf วันที่}   & {\bf หน้าที่}                    & {\bf จุดแก้ไข}          & {\bf บันทึกการแก้ไข} \\ \hline
        2 ธันวาคม    & \pageref{sec:introduction}    & สถิต หรือ สถิตย์          & ตรวจสอบจากพจานานุกรมฉบับสำนักงานราชบัณฑิตยสภาเรียบร้อยแล้ว 
                                                            ใช้คำว่า {\bf "สถิต"} ส่วนคำว่า {\bf "สถิตย์"} ไม่มีในพจนานุกรมครับ \\ \hline
                    &  \pageref{sec:introduction}   & ไม่พบปัญหาที่เป็นประเด็น    & ปรับบทนำ \\ \hline
                    & -                             & รูปที่ x (หน้า n)         & ตัดคำว่า\ {\bf (หน้า n)}\ ทิ้งไป \\ \hline
                    & 1-2                           & ปัญหาและความสำคัญไม่ชัดเจน& เพิ่มคำอธิบายลักษณะการทำงาน ปัญหาที่พบ 
                                                            สมมติฐาน และสิ่งที่ต้องการนำเสนอ \\ \hline
                    & \pageref{sec:sub:sub:cfg}     & ปรับเนื้อหา              & ปรับเนื้อหาให้กระชับมากขึ้น \\ \hline
                    & \pageref{sec:sub:sub:scg}     & ปรับเนื้อหา              & ปรับเนื้อหาให้กระชับมากขึ้น \\ \hline
                    & \pageref{sec:sub:infeasible-path}     & ปรับเนื้อหา              & ปรับเนื้อหาให้กระชับมากขึ้น \\ \hline
                    & \pageref{sec:related-work}    & ปรับเนื้อหา              & เพิ่มเนื้อหางานวิจัย \\ \hline
                    & \pageref{sec:sub:bandb}       & เพิ่มงานวิจัยที่เกี่ยวข้อง     & เพิ่มงานวิจัยที่เกี่ยวข้องกับการสร้างกรณีทดสอบด้วย{\Algorithm} 
                                                            Branch and bound\\ \hline
                    & \pageref{sec:objective}       & ปรับวัตถุประสงค์ ข้อ 1.    & ยกประโยค {\textit "เพื่อสร้างกรณีทดสอบ..."} 
                                                            ไปใส่ในขอบเขต \\ \hline 
                    & \pageref{sec:objective}       & ปรับวัตถุประสงค์ ข้อ 2.    & ปรับเป็นการสร้างเครื่องมือตามแนวคิดในข้อ 1. \\ \hline
                    & \pageref{sec:limitation}      & ปรับข้อความใน ข้อ \ref{enu:lim:datatype}.    & เปลี่ยน {\it Sequencial programming} 
                                                            เป็น {\it Parallel} \\ \hline
                    & \pageref{sec:limitation}      & ปรับข้อความใน ข้อ \ref{enu:lim:seq}.         & ปรับข้อความเป็นการสร้างเครื่องมือตามแนวคิดในข้อ 1. \\ \hline
                    & \pageref{sec:limitation}      & ปรับข้อความใน ข้อ \ref{enu:lim:loop}.        & ลดจำนวนรอบลงจาก 2 เป็น 1 รอบ \\ \hline
    \end{tabularx}
\end{table}
