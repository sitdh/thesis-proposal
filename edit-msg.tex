\pagenumbering{thaialph}
\begin{table}[ht!]
    \caption{รายการแก้ไขจากความเห็น รศ.ดร. พรศิริ หมื่นไชยศรี}
    \centering
    \begin{tabular} % {\textwidth}{|l|l|l|X|}
        \hline
        \rowcolor{LightGray} 
        {\bf รายการ}    & {\bf หัวข้อ}                            & {\bf จุดแก้ไข}          & {\bf บันทึกการแก้ไข} \\ \hline
        1.              & วิธีนี้รับประกันการครอบคลุมใด มากน้อยแค่ไหน    & ข้อที่ \ref{enu:lim:branchcoverage} จาก ขอบเขตงานวิจัย                   & เพิ่มเติมข้อกำหนด \\ \hline
        2.              & ต่างจาก Symbolic execution อย่างไร      
                        & -                     
                        & ไม่ต่างครับ เพราะเป็นการนำเอา Symbolic execution ที่อยู่หลายคลาสบนทางเดินทดสอบมาพิจารณาร่วมกัน \\ \hline
        3.              & งานนี้ทำงานระดับ Unit testing หรือ Integration testing ทำเพียง 1 คลาส หรือไม่ 
                                                                & -
                                                                & ทำระดับ Integration Testing ที่หลายคลาสทำงานร่วมกัน \\ \hline
        4.              & ผลลัพธ์มีอะไรบ้าง                         & -                     & กรณีทดสอบ \\ \hline
        5.              & ปรับตัวอักษรใหญ่แต่ละคำ                    & หัวข้อภาษาอังกฤษ         & ปรับเป็นตัวอักษรใหญ่เมื่อเริ่มคำ \\ \hline 
        6.              & อธิบายการศึกษาโครงสร้าง (ชื่อเดิม สร้างกราฟ) & ข้อ 1 และ 2 หัวข้อ 4.1.2 & เพิ่มเนื้อคำอธิบายวิธีการสร้างกรฟา \\ \hline
        7.              & แก้คำผิด                                & ข้อ 4.2.2              & แก้ไขคำว่า SATISFIED ให้ถูกต้อง \\ \hline
        8.              & รองรับลูปหรือไม่                          & ขอบเขต ข้อ 7 หัวข้อที่ 8   & เพิ่มข้อความรองรับ Loop \\ \hline
        9.              & ใช้กราฟมาช่วยในข้อง 4.2.2 อย่างไร         
                        & -                     
                        & เพื่อให้ทราบว่ามีเงื่อนไขใดบ้างที่อยู่บนเส้นทางทดสอบที่เลือกมาจากข้อ 4.2.1 ตลอดจนชนิดของรายการข้อมูลนำเข้าที่ได้จากรายการพารามิเตอร์ของ method signature 
                          ท้ายที่สุดนั้น ช่วยให้สามารถตัดรายการของเงื่อนไขที่ไม่จำเป็นต่อการสร้างกรณีทดสอบออกไป 
                          ในกรณีที่รายการตัวแปรบนเงื่อนไขนั้นไม่มีความสัมพันธ์เกี่ยวข้องใดๆ กับรายการข้อมูลนำเข้า\\ \hline
    \end{tabular}
\end{table}

\begin{table}[ht!]
    \caption{รายการแก้ไขจากความเห็น รศ.ดร. วิวัฒน์ วัฒนาวุฒิ}
    \centering
    \begin{tabularx}{\textwidth}{|l|l|l|X|}
        \hline
        \rowcolor{LightGray} 
        {\bf รายการ}    & {\bf หัวข้อ}                            & {\bf จุดแก้ไข}          & {\bf บันทึกการแก้ไข} \\ \hline
        1.              & นิยาม Call graph ยังไม่ชัดเจน             & ข้อ 2.1.2 และข้อ 1 หัวข้อ 4.1.2       & เพิ่มนิยามและวิธีการสร้าง \\ \hline
    \end{tabularx}
\end{table}

\begin{table}[ht!]
    \caption{รายการแก้ไขจากความเห็น ผศ.ดร. อาทิตย์ ทองทักษ์}
    \centering
    \begin{tabularx}{\textwidth}{|l|l|l|X|}
        \hline
        \rowcolor{LightGray} 
        {\bf รายการ}    & {\bf หัวข้อ}                            & {\bf จุดแก้ไข}          & {\bf บันทึกการแก้ไข} \\ \hline
        1.              & แก้ไขคำผิด                              & หน้าปก                 & {\bf "วิทยาศาสตร์มหาบัณฑิต"} เป็น {\bf "วิทยาศาสตรมหาบัณฑิต"} \\ \hline
        2.              & ขอบเขตข้อ 7, 8, 10 (ก่อนแก้ไข) เป็นข้อจำกัดของกราฟหรืออะไร      
                        & -                   
                        & - ข้อ 7 เป็นข้อจำกัดของกราฟและการสร้างข้อมูลทดสอบ, \newline 
                          - ข้อ 8 เนื่องจากใช้วิธีการสร้างกรณีทดสอบโดยพิจารณาจากเงื่อนไขที่พบบนทางเดินทดสอบ หากพบเงื่อนไขมากหรือซับซ้อนจะส่งผลต่อประสิทธิภาพการทำงาน \newline  
                          - ข้อ 10 กราฟไม่สามารถจำลองโครงสร้างแบบ Recursive ได้ จึงละเว้นไว้ อีกทั้งงานวิจัยชิ้นนี้เน้นการทดสอบเพื่อค้นหาข้อผิดพลาดที่เกิดจากการเชื่อมต่อกันระหว่างคลาส \\ \hline
    \end{tabularx}
\end{table}

\begin{table}[ht!]
    \caption{รายการแก้ไขอื่นๆ}
    \centering
    \begin{tabularx}{\textwidth}{|l|l|l|X|}
        \hline
        \rowcolor{LightGray} 
        {\bf รายการ}    & {\bf หัวข้อ}                                            & {\bf จุดแก้ไข}          & {\bf บันทึกการแก้ไข} \\ \hline
        1.              & ตัดขอบเขตข้อ 2 และ 3 ทิ้งไปเนื่องจากยังหาจำนวนที่ชัดเจนไม่ได้     & ขอบเขตงานวิจัย          & ลบข้อความ \\ \hline
        1.              & ขอบเขตให้ชัดเจนยิ่งขึ้น                                     & ข้อ 1 หัวข้อที่ 6          & แก้ไขข้อความ  \\ \hline
        2.              & ขอบเขตข้อ 7, 8, 10 (ก่อนแก้ไข) เป็นข้อจำกัดของกราฟหรืออะไร      & xxx                   & eeee \\ \hline
    \end{tabularx}
\end{table}
