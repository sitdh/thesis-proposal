\pagenumbering{thaialph}
\begin{table}[ht!]
    \caption{รายการแก้ไขจากความเห็น รศ.ดร. พรศิริ หมื่นไชยศรี}
    \centering
    \begin{tabularx}{\textwidth}{|l|l|l|X|}
        \hline
        \rowcolor{LightGray} 
        {\bf รายการ}    & {\bf หัวข้อ}                            & {\bf จุดแก้ไข}          & {\bf บันทึกการแก้ไข} \\ \hline
        1.              & วิธีนี้รับประกันการครอบคลุมใด มากน้อยแค่ไหน    & สถิต หรือ สถิตย์          & ตรวจสอบจากพจานานุกรมฉบับสำนักงานราชบัณฑิตยสภาเรียบร้อยแล้ว \\ \hline
        2.              & ต่างจาก Symbolic execution อย่างไร      & xxx                   & eeee \\ \hline
        3.              & งานนี้ทำงานระดับ Unit testing หรือ Integration testing ทำเพียง 1 คลาส หรือไม่ 
                                                                & xxx                   & xxxxx \\ \hline
        4.              & ผลลัพธ์มีอะไรบ้าง                         & xxx                   & xxxx \\ \hline
        5.              & ปรับตัวอักษรใหญ่แต่ละคำ                    &                       &           \\ \hline 
        6.              & อธิบายการศึกษาโครงสร้าง (ชื่อเดิม สร้างกราฟ) & xxx                   & xxx \\ \hline
        7.              & แก้คำผิด                                & xxx                   & yyy \\ \hline
        8.              & รองรับลูปหรือไม่                          & xxx                   & yyy \\ \hline
        9.              & ใช้กราฟมาช่วยในข้อง 4.2.2 อย่างไร         & xxx                   & yyy \\ \hline
    \end{tabularx}
\end{table}

\begin{table}[ht!]
    \caption{รายการแก้ไขจากความเห็น รศ.ดร. วิวัฒน์ วัฒนาวุฒิ}
    \centering
    \begin{tabularx}{\textwidth}{|l|l|l|X|}
        \hline
        \rowcolor{LightGray} 
        {\bf รายการ}    & {\bf หัวข้อ}                            & {\bf จุดแก้ไข}          & {\bf บันทึกการแก้ไข} \\ \hline
        1.              & นิยาม Call graph ยังไม่ชัดเจน             & xxx                   & yyy \\ \hline
        2.              & ขอบเขตข้อ 7, 8, 10 (ก่อนแก้ไข) เป็นข้อจำกัดของกราฟหรืออะไร      & xxx                   & eeee \\ \hline
    \end{tabularx}
\end{table}

\begin{table}[ht!]
    \caption{รายการแก้ไขจากความเห็น ผศ.ดร. อาทิตย์ ทองทักษ์}
    \centering
    \begin{tabularx}{\textwidth}{|l|l|l|X|}
        \hline
        \rowcolor{LightGray} 
        {\bf รายการ}    & {\bf หัวข้อ}                            & {\bf จุดแก้ไข}          & {\bf บันทึกการแก้ไข} \\ \hline
        1.              & แก้ไขคำผิด                              & หน้าปก                 & {\bf "วิทยาศาสตร์มหาบัณฑิต"} เป็น {\bf "วิทยาศาสตรมหาบัณฑิต"} \\ \hline
        2.              & ขอบเขตข้อ 7, 8, 10 (ก่อนแก้ไข) เป็นข้อจำกัดของกราฟหรืออะไร      & xxx                   & eeee \\ \hline
    \end{tabularx}
\end{table}
