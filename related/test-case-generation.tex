\subsection{การสร้างกรณีทดสอบอัตโนมัติ (Automated test case generation)}
\label{sub:tcgen}

% - อ้างอิงจาก "An orchestrated survey of methodologies for automated software test case generation" \cite{Anand2013} Part II
กระบวนการทดสอบซอฟต์แวร์นั้นสามารถสร้างกรณีทดสอบอัตโนมัติ ซึ่ง Anand และคณะ \cite{Anand2013} ได้แบ่งวิธีการสร้างกรณีทดสอบแบบอัตโนมัติไว้ทั้งหมด 
4 กลุ่มวิธีการด้วยกัน ซึ่งมี 2 กลุ่มวิธีการที่เกี่ยวข้องได้แก่

\subsubsection{Symbolic execution}
\label{sub:tcgen:sub:symbolic}

วิธีการนี้เป็นกระบวนการสร้างกรณีทดสอบที่ใช้การวิเคราะห์ชุดคำสั่งควบคุมการไหลของโปรแกรมแล้วสร้างเป็นเงื่อนไขของทางเดิน 
ใช้วิเคราะห์หาข้อมูลที่สามารถทำให้กรณีทดสอบนั้นสามารถทดสอบทางเดินที่เลือกได้ ซึ่งวิธีนี้มีประสิทธิภาพเรื่องความครอบคลุมของ{\sourcecode}ได้เป็นอย่างดี
หากแต่ข้อเสียที่พบคือ โปรแกรมที่ใช้งานจริงนั้นมักจะมีเงื่อนไขหลากหลายและซับซ้อนเกินกว่าจะสร้างข้อมูลทดสอบอย่างอัตโนมัติได้ทั้งหมด 
นอกจากนั้นยังจำเป็นจะต้องอาศัยการตัดสินใจจากผู้ใช้งานในบางกรณี

\subsubsection{Random testing}
\label{sub:tcgen:sub:random}

จากศึกษาพบว่าข้อมูลที่ทำให้เกิดข้อผิดพลาดนั้นมีแนวโน้มที่จะอยู่รวมกันเป็นรูปแบบ %ดัง{\figref{fig:failureRegionPattern}} 
ดังนั้นกรณีทดสอบที่สร้างขึ้นควรจะต้องกระจายให้ครอบคลุมทั้งกลุ่มข้อมูลนำเข้า (Input domain) เพิ่มโอกาสการค้นหาข้อผิดพลาดภายในโปรแกรม 
ซึ่งมีแนวทาง Adaptive Random Testing โดย Chan และคณะ \cite{Chan2004}\ ที่พัฒนาขึ้นเพื่อเพิ่มประสิทธิภาพของ Random testing ในรูปแบบเดิม 
หากแต่ปัญหาของการทำ Random testing นั้นเกิดมาจากการต้องการสุ่มค่านั้นเอง 
เพราะมีโอกาสที่ค่าที่สุ่มขึ้นมานั้นมีจำนวนมากและไม่สามารถค้นพบข้อผิดพลาดที่อยู่ภายในโปรแกรม ดังนั้นวิธีการหนึ่งที่มักจะอ้างถึงใน Adaptive Random Testing 
นั้นก็คือการกำหนดกลุ่มข้อมูลที่เป็นไปได้ (Fixed-Sized-Candidate-Set: FSCS-ART) ของ Adaptive Random Testing 
ด้วยการกำหนดค่าสูงสุดหรือต่ำสุดที่ต้องการสุ่มค่าออกมากได้ กำหนดรูปแบบหรือประเภทของข้อมูลที่ต้องการสุ่ม 
เพื่อให้ค่าที่สุ่มขึ้นมานั้นมีโอกาสในการค้นพบข้อผิดพลาดมากขึ้น % ใกล้เคียงกับความต้องการมากที่สุด
